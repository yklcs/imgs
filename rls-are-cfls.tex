\documentclass{imgs}

\begin{document}

\begin{theorem}
  All regular languages are context-free languages.
  \begin{proof}
    Let $L$ be a regular language, and $R$ be the regular expression such that $L = L(R)$.
    We perform induction on the number of operators in $R$.
    \begin{description}
      \item[Basis] $R$ has no operators. This means $R = a$ is a single symbol.
            Then, the language can be generated with a CFG using a single transition $S \to a$.
      \item[Induction]
            Suppose we can represent any regular expression with less than $k$ operators with a CFG.
            Now we consider the case where $R$ has $n$ operators.

            Let $R_1$ and $R_2$ the operand REs of $R$.
            By the inductive hypothesis, we can represent $R_1$ and $R_2$ with CFGs with start states $S_1$ and $S_2$, respectively.

            \begin{itemize}
              \item $R = R_1 + R_2$:
                    We can create a new CFG with start state $S$ with the production $S \to S_1 \mid S_2$.
                    This produces $L(R_1 + R_2)$.
              \item $R = R_1 R_2$:
                    We can create a new CFG with start state $S$ with the production $S \to S_1 S_2$.
                    This produces $L(R_1 R_2)$.
              \item $R = R_1^*$:
                    We can create a new CFG with start state $S$ with the production $S \to S S_1 \mid \varepsilon$.
                    This produces $L(R_1^*)$.
            \end{itemize}
    \end{description}
    We can represent any RE with an equivalent CFG.
    Therefore, every regular language is context-free.
  \end{proof}
\end{theorem}
\end{document}

%#automata
%#cfl
%#rl
